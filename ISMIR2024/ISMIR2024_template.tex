% -----------------------------------------------
% Template for ISMIR Papers
% 2023 version, based on previous ISMIR templates

% Requirements :
% * 6+n page length maximum
% * 10MB maximum file size
% * Copyright note must appear in the bottom left corner of first page
% * Clearer statement about citing own work in anonymized submission
% (see conference website for additional details)
% -----------------------------------------------

% ISMIR REVIEW CRITERION
% https://tomcollinsresearch.net/pdf/ismirReviewExamples.pdf

\documentclass{article}
\usepackage[T1]{fontenc} % add special characters (e.g., umlaute)
\usepackage[utf8]{inputenc} % set utf-8 as default input encoding
\usepackage{ismir,amsmath,cite,url}
\usepackage{graphicx}
\usepackage{color}


\usepackage{lineno}
\linenumbers

% Title. Please use IEEE-compliant title case when specifying the title here,
% as it has implications for the copyright notice
% ------
\title{Synthesizing Flexible, Composite Hierarchical Structure from Music Datasets}

% Note: Please do NOT use \thanks or a \footnote in any of the author markup

% Single address
% To use with only one author or several with the same address
% ---------------
\oneauthor
	{Ilana Shapiro}
	{UC San Diego \\ {\tt ilshapiro@ucsd.edu}}
 %{Names should be omitted for double-blind reviewing}
 %{Affiliations should be omitted for double-blind reviewing}

% Two addresses
% --------------
%\twoauthors
%  {First author} {School \\ Department}
%  {Second author} {Company \\ Address}

% Three addresses
% --------------\input{ISMIR2021_paper.tex}

%\threeauthors
%  {Ilana Shapiro} {Affiliation1 \\ {\tt author1@ismir.edu}}
%  {Second Author} {\bf Retain these fake authors in\\\bf submission to preserve the formatting}
%  {Third Author} {Affiliation3 \\ {\tt author3@ismir.edu}}

% Four or more addresses
% OR alternative format for large number of co-authors
% ------------
%\multauthor
%{First author$^1$ \hspace{1cm} Second author$^1$ \hspace{1cm} Third author$^2$} { \bfseries{Fourth author$^3$ \hspace{1cm} Fifth author$^2$ \hspace{1cm} Sixth author$^1$}\\
%  $^1$ Department of Computer Science, University , Country\\
%$^2$ International Laboratories, City, Country\\
%$^3$  Company, Address\\
%{\tt\small CorrespondenceAuthor@ismir.edu, PossibleOtherAuthor@ismir.edu}
%}

% For the author list in the Creative Common license, please enter author names. 
% Please abbreviate the first names of authors and add 'and' between the second to last and last authors.
\def\authorname{Ilana Shapiro}

% Optional: To use hyperref, uncomment the following.
%\usepackage[bookmarks=false,pdfauthor={\authorname},pdfsubject={\papersubject},hidelinks]{hyperref}
% Mind the bookmarks=false option; bookmarks are incompatible with ismir.sty.

\sloppy % please retain sloppy command for improved formatting

\begin{document}

%
\maketitle
%

%We adopt a ``(6+n)-page policy'' for ISMIR \conferenceyear. That is, each paper may have a maximum of six pages of technical content (including figures and tables) \textcolor{red}{with additional optional pages that contain only references and acknowledgments. Note that acknowledgments should not be included in the anonymized submission.}
%Paper should be submitted as PDFs and the \textcolor{red}{file size is limited to 10MB}. Please compress images and figures as necessary before submitting.

\begin{abstract}
Music structure analysis is an open research problem in the MIR community. Analyses for However, the problem of how to combine
How to relate their results both semantically (i.e. informed by musical theory) and temporally (i.e. informed by the timestamps of the structure labels)?
Also, how to find a representative structure for a corpus of pieces, rather than a single piece?
Goal: using existing structure analyses, abstract them into a data structure (graph) that combines them. Then, average these graphs into a “centroid” that represents the whole corpus

\end{abstract}
%
\section{Introduction}\label{sec:intro}

\cite{MIREX_2017_motif}
\section{Related work}\label{sec:related_work}

\section{Analysis Formats}\label{sec:formats}
\subsection{MIREX Standard Formats} \label{subsec:MIREX}
\subsection{Parsing} \label{subsec:parsing}

\section{Abstract Representation}\label{subsec:representation}

\subsection{Semantic Temporal Graph} \label{subsec:st_graph}

\section{Synthesis} \label{subsec:synthesis}

%\section{Evaluation} \label{subsec:eval}

\section{Conclusions and Future Work}

All bibliographical references should be listed at the end of the submission, in a section named ``REFERENCES,''
numbered and in the order that they first appear in the text. Formatting in the REFERENCES section must conform to the
IEEE standard (\url{https://ieeeauthorcenter.ieee.org/wp-content/uploads/IEEE-Reference-Guide.pdf}). Approved
IEEE abbreviations (Proceedings $\rightarrow$ Proc.) may be used to shorten reference listings. All references listed
should be cited in the text. When referring to documents, place the numbers in square brackets (e.g., \cite{ISMIR17Author:01}
for a single reference, or \cite{JNMR10Someone:01,Book20Person:01,Chapter09Person:01} for a range).


%\section{Acknowledgments}
%\textbf{Do not include in your submission, only in your camera ready version}. This section can be used to refer to any individuals or organizations that should be acknowledged in this paper. This section does \textit{not} count towards the page limit for scientific content.


% For bibtex users:
% https://archives.ismir.net/ismir2023/paper/000045.pdf (new ISMIR paper on hierarchical grammars)
% https://archives.ismir.net/ismir2014/paper/000226.pdf (previous work on grammar extraction for form)
% https://archives.ismir.net/ismir2019/paper/000039.pdf (grammar for Harmony)
% https://citeseerx.ist.psu.edu/document?repid=rep1&type=pdf&doi=a602b8d972f68b100350c5dc54681cc44f31493a (sCluster)
% https://www.ee.columbia.edu/~dpwe/pubs/McFeeE14-segments.pdf (OLDA)
% https://www.frontiersin.org/journals/psychology/articles/10.3389/fpsyg.2017.01337/full (evaluating hierarchical analyses)
% https://ccrma.stanford.edu/~urinieto/MARL/publications/Nieto-Dissertation.pdf (MSAF)
% https://ccrma.stanford.edu/~urinieto/MARL/publications/NietoBello-ICASSP14.pdf (2D-FMC)
% https://cacm.acm.org/research/stochastic-program-optimization/ (stochastic program optimization)
% https://escholarship.org/uc/item/7tg5c8rb / https://www.semanticscholar.org/paper/Structural-segmentation-with-the-Variable-Markov-Wang-Mysore/1f9ac4bb9527d47deec00f24efd72d7c1796ca0c / https://www.ismir2015.uma.es/articles/78_Paper.pdf(VMO)
% https://ismir2023program.ismir.net/poster_145.html (motif extraction)
% https://proceedings.neurips.cc/paper_files/paper/2022/hash/f13ceb1b94145aad0e54186373cc86d7-Abstract-Conference.html (halley, prototype graph)
% https://arxiv.org/pdf/2209.07974.pdf / https://www.sciencedirect.com/science/article/pii/S2352711023000614 (musicaiz)
% https://hal.science/hal-03278537/ (morpheauS)
% https://ismir2018.ircam.fr/doc/pdfs/178_Paper.pdf (harmony 1)
% https://archives.ismir.net/ismir2019/paper/000030.pdf (harmony 2)
% https://arxiv.org/pdf/2107.05223.pdf (melody 1)
% https://archives.ismir.net/ismir2022/paper/000091.pdf (melody 2)
% https://www.justinsalamon.com/uploads/4/3/9/4/4394963/jsalamon_phdthesis.pdf (melody 3)
% https://colinraffel.com/publications/ismir2014mir_eval.pdf (mir eval)


\bibliography{ISMIRtemplate}

% For non bibtex users:
%\begin{thebibliography}{citations}
% \bibitem{Author:17}
% E.~Author and B.~Authour, ``The title of the conference paper,'' in {\em Proc.
% of the Int. Society for Music Information Retrieval Conf.}, (Suzhou, China),
% pp.~111--117, 2017.
%
% \bibitem{Someone:10}
% A.~Someone, B.~Someone, and C.~Someone, ``The title of the journal paper,''
%  {\em Journal of New Music Research}, vol.~A, pp.~111--222, September 2010.
%
% \bibitem{Person:20}
% O.~Person, {\em Title of the Book}.
% \newblock Montr\'{e}al, Canada: McGill-Queen's University Press, 2021.
%
% \bibitem{Person:09}
% F.~Person and S.~Person, ``Title of a chapter this book,'' in {\em A Book
% Containing Delightful Chapters} (A.~G. Editor, ed.), pp.~58--102, Tokyo,
% Japan: The Publisher, 2009.
%
%
%\end{thebibliography}

\end{document}

